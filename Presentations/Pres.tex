% Copyright 2004 by Till Tantau <tantau@users.sourceforge.net>.
%
% In principle, this file can be redistributed and/or modified under
% the terms of the GNU Public License, version 2.
%
% However, this file is supposed to be a template to be modified
% for your own needs. For this reason, if you use this file as a
% template and not specifically distribute it as part of a another
% package/program, I grant the extra permission to freely copy and
% modify this file as you see fit and even to delete this copyright
% notice. 


\documentclass{beamer}


\usepackage{hyperref}
\usetheme{Warsaw}

\title{Presentation}


\author{L.~Meyer \and D.~Srage}


\date{Presentation at the University of Chicago, Februray 2015}
% - Either use conference name or its abbreviation.
% - Not really informative to the audience, more for people (including
%   yourself) who are reading the slides online


% If you have a file called "university-logo-filename.xxx", where xxx
% is a graphic format that can be processed by latex or pdflatex,
% resp., then you can add a logo as follows:

 \pgfdeclareimage[height=1cm]{university-logo}{Logo_ponts_paristech.png}
 \logo{\pgfuseimage{university-logo}}

% Delete this, if you do not want the table of contents to pop up at
% the beginning of each subsection:
\AtBeginSubsection[]
{
  \begin{frame}<beamer>{}
    \tableofcontents[currentsection,currentsubsection]
  \end{frame}
}

% Let's get started
\begin{document}

\begin{frame}
  \titlepage
\end{frame}

\begin{frame}{Outline}
  \tableofcontents
  % You might wish to add the option [pausesections]
\end{frame}

% Section and subsections will appear in the presentation overview
% and table of contents.
\section{Value of an European option}

\subsection{Basic concepts}

%first frame
\begin{frame}{What is an European value ?}
\begin{definition}
\textbf{European option} : In finance, it is a contract which gives the buyer (the owner) the right, to buy or sell a stock at a specified \alert{strike price} on a specified date. This date \emph{T} is called the \alert{maturity} of the option. 
\end{definition}
We must also give the following definitions :
\begin{itemize}
\item \alert{Stock price} : Current price of the stock.
\item \alert{Spot price} : Expected price of the stock on a given date.
\end{itemize}
\end{frame}

%second frame
\begin{frame}{The value of an option}
An option can be exchanged on the market like a stock. Therefore, there is an option value. This value is defined by a relation in which the spot price intervenes.
\end{frame}

\subsection{Underlying assumptions}
%third frame
\begin{frame}{Assumptions}
\begin{itemize}
\item We consider the Black-Scholes model.
\item The characteristic function of the log terminal sport price is known.
\item Initially we have $S_{0}=1$
\item The interest rate is constant
\end{itemize}
\end{frame}


\section{Calculus of the option value using FFT}

\subsection{Expression of the option value}

\begin{frame}{Initial call value}
\begin{block}{Definition of the call value}
\begin{equation}
C_{T}(k)=\int_{k}^{\infty}e^{-rT}(e^{s}-e^{k})q_{T}(s)ds
\end{equation}
\end{block}
\begin{itemize}
\item $q_{T}$ the risk-neutral density of the log spot price
\item \emph{k} the log stike price of the stock
\item \emph{r} the interest rate
\end{itemize}
\end{frame}

\begin{frame}{Transform of the call value}
As we want a square-integrable function we consider
\begin{equation*}
c_{T}(k)=exp(\alpha k)C_{T}(k)
\end{equation*}
That leads to the following equation
\begin{block}{More useful expression of call value}
\begin{equation}
C_{T}(k)=\frac{exp(-\alpha k)}{\pi}\int_{0}^{\infty}e^{-i \nu k}\psi(\nu)d\nu
\label{eq:cv}
\end{equation}
\end{block}
Where $\psi$ is the Fourier transform of $c_{T}$.
\end{frame}

\begin{frame}{The purpose of the FFT}
We would like to use the FFT to compute quickly the call value given by the expression \eqref{eq:cv}. In order to do that, we should write $C_{T}(k)$ like the following sum
\begin{equation*}
\sum_{j=1}^{N}e^{-i\frac{2\pi}{N}(j-1)(k-1)}x(j)
\end{equation*}
To obtain a such expression, we try two different methods. First, we approach the integral in \eqref{eq:cv} by the trapezoid method. Then we will consider the Simpson's method.
\end{frame}

\subsection{Trapezoid rule}

\begin{frame}{Trapezoid rule}
\begin{block}{Final result for $C_T(k)$}
\begin{equation}
C_T(k) \approx \frac{exp(- \alpha k)}{\pi}\sum_{i=1}^N e^{-i\nu_jk} \psi_T(\nu_j)\eta
\end{equation}
\end{block}

Demonstration:
\newline
\only<1-2>{
\begin{equation*}
C_T(k)=\lim_{a \to\infty} \frac{exp(- \alpha k)}{\pi}\int_{0}^a e^{-i \nu k}\psi_T(\nu)d\nu
\end{equation*}
}
\only<2>{
Thanks to the Trapezoidal rule, we can write:
\begin{align*}
C_T(k)&=\lim_{a \to\infty} \frac{exp(- \alpha k)}{\pi} \frac{a}{N}\sum_{j=2}^N e^{-i k (j-1)\frac{a}{N}}\psi_T((j-1)\frac{a}{N}) \\
&+\frac{a}{2N}(\psi_T(0)+e^{-ika}\psi_T(a))
\end{align*}
}
\only<3->{
And because $\frac{a}{N}=\eta$ and $\lim_{a \to \infty} \psi_T(a)=0$ , we have:
\begin{equation*}
C_T(k)= \frac{exp(- \alpha k)}{\pi} \frac{a}{N}\sum_{i=1}^N e^{-i\nu_jk} \psi_T(\nu_j)\eta +\epsilon
\end{equation*}
}
\end{frame}


\subsection{Simpson's rule}
\begin{frame}{Simpson's rule}
\begin{block}{Final result for $C_T(k)$}
\begin{equation}
C_T(k) \approx \frac{exp(- \alpha k)}{\pi}\sum_{i=1}^N e^{-i\nu_jk} \psi_T(\nu_j)\frac{\eta}{3}(3+(-1)^j-\delta_{j-1})
\end{equation}
\end{block}
Demonstration:
\newline
\only<1>{
Thanks to the Simpson's rule, we get
\begin{equation*}
C_T(k)=\lim_{a \to\infty} \frac{exp(- \alpha k)}{\pi}\int_{0}^a e^{-i \nu k}\psi_T(\nu)d\nu
\end{equation*}
}
\only<2>{
\begin{align*}
\int_{0}^a e^{-i \nu k}\psi_T(\nu)d\nu&=\frac{2\eta}{6}\sum_{j=0}^{N-1}e^{-ik2j\frac{\eta}{2}}\psi_T(2j\frac{\eta}{2})\\
&+\frac{4\eta}{6} \sum_{j=0}^{N-1} e^{-ik(2j+1)\frac{\eta}{2}}\psi_T{(2j+1)\frac{\eta}{2}}\\
&+\frac{\eta}{6}(\psi_{T}(0)+e^{-ika}\psi_{T}(a))
\end{align*}}
\only<3->{
Then we replace $\frac{\eta}{2}$ with $\eta'$ and therefore $N$ by $\frac{N'}{2}$, and we get the result by defining $\epsilon=\frac{\eta}{2}\psi_T(0)$
}
\end{frame}
% All of the following is optional and typically not needed. 
\section{Future goals}
\begin{frame}{Future goals}
\begin{itemize}
\item We can main compute the call value using the FFT.
\item We can compare the fact that Simpson's rule is more powerful the Trapezoidal rule.
\item We will use this two different rules to evaluate the call value with different characteristic functions, for instance the \alert{Black-Scholes} model or the \alert{VG} model, and we will compare them to each other and to the result we have when we calculate the Fourier transform of the different models. 
\end{itemize}
\end{frame}

\appendix
\section<presentation>*{\appendixname}
\subsection<presentation>*{Bibliography}

\begin{frame}[allowframebreaks]
  \frametitle<presentation>{Bibliography}
    
  \begin{thebibliography}{10}
   
    
  \beamertemplatearticlebibitems
  % Followed by interesting articles. Keep the list short. 

  \bibitem{FirstArt}
   P.~Carr  D.~Madan
   \newblock Option valuation using the fast Fourier transform
   \newblock{\em Journal of computational finance}

  \bibitem{SecondtArt}
    M.~Attari.
    \newblock Option pricing using Fourier transform : a numercially efficient simplification
    \newblock March 21, 2004
  \end{thebibliography}
\end{frame}

\end{document}
